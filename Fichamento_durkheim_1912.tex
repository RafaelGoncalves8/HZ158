\documentclass[12pt,a4paper]{article}
\setlength\parindent{24pt}
\usepackage[margin=2.5cm]{geometry}
\usepackage[utf8]{inputenc}
\usepackage[portuguese]{babel}
\renewcommand{\baselinestretch}{1.5}
\usepackage[bottom]{footmisc}

\begin{document}

\noindent
\centerline{\large\textbf{HZ158 | Fichamento d'As formas elementares da vida religiosa}}\footnote{DURKHEIM, Émile. 1996. As formas elementares da vida religiosa. (Trad. Paulo Neves) São Paulo: Martins Fontes [1912]}
\centerline{\large Rafael Gonçalves (186062)}
\break
\noindent
O autor inicia a obra expondo seu objeto de pesquisa: as religiões das sociedades primitivas (ou ainda como o título explicita: as formas elementares da vida religiosa). Para analisar a instituição religiosa nessas sociedades, Durkheim faz uso de etnografias - que seriam o signo visível, concreto e analisável do objeto a ser estudado. Diferentemente de seus livros anteriores - muito mais focados nos modos de pensar e de agir da sociedade - os relatos etnograficos possibilitam a análise de uma esfera mais emocional, intensiva, ou os modos de sentir.

Por mais que tenham significados diferentes, Durkheim usa as palavras \em primitivo\em , \em simples\em , \em elementar \em e \em original \em quase que como sinônimos ao longo do livro. Isso mostra uma característica importante do pensamento durkheimiano, a saber que as primeiras sociedades (originais, de onde se origina a vida em sociedade) seriam as sociedades mais simples, com menor grau de evolução (primitivas) e também elementares - as que conteriam os elementos essenciais para se entender as sociedades posteriores. Com isso podemos perceber como há no autor uma noção de evolução \em natural \em em uma direção específica que parte do simples, primitivo, homogêneo e elementar rumo ao complexo, civilizado, heterogêneo e diferenciado.

Dessa forma Durkheim busca analisar a vida religiosa por meio do estudo das sociedades em que ela se mostra da maneira mais simples, mas que ainda assim conteria os elementos para entender sua expressão mesmo nas sociedades de sua época tidas como mais complexas e evoluídas. Mais do que analisar a religião em sua forma primitiva, Durkheim busca analisar a própria origem da instituição religiosa, ou ainda a origem da própria vida em sociedade (\em sociogênese\em ).

Para o autor, a religião é essencialmente \em coisa social\em, pois
``as representações religiosas são representações coletivas que exprimem realidades coletivas; os ritos são maneiras de agir que só surgem no interior de grupos coordenados e se destinam a suscitar, manter ou refazer alguns estados mentais desses grupos." (p. XVI).
Ao tentar entender a origem da vida religiosa - primeira instituição cronologicamente, mas também de onde partiriam todas as outras -, Durkheim busca entender a origem da vida em sociedade. Como os indivíduos se associam a fim de formar uma sociedade.

Então o autor segue fazendo uma definição mais precisa do que seria a religião. Durkheim argumenta que para definir a religião é preciso definir seu sistema de representações (crenças) e suas práticas (ritos) (p. 19) . Todo conjunto de crenças parte da cassificação das coisas em dois gêneros opostos, de naturezas distintas, que Durkheim generaliza nos termos \em sagrado \em e \em profano\em . A religião nasce da experiência do sagrado, momento de comunhão, associação de indivíduos, nascimento do sagrado. Logo, em contraposição, aparece também o profano, o não-sagrado, o trivial, o puramente individual.
\begin{quote}
    " (...) não nos resta outra coisa para definir o sagrado em relação ao profano, a não ser sua heterogeneidade. E o que torna essa heterogeneidade suficiente para caracterizar semelhante classificação das coisas e distingui-la de qualquer outra é justamente o fato de ela ser muito particular: \em ela é absoluta\em ." (p. 21-22)
\end{quote}
Ante a essa classificação, \em os indivíduos se unem em favor do sagrado\em , tentam repetir a experiência do mesmo, institucionalizá-la. As crenças serão o que exprime a natureza do sagrado e sua relação com o profano, enquanto que os ritos serão os modos de agir ante as coisas sagradas. Esse sistema de crenças e ritos é o que constitue uma religião (p. 24).

É enfatizado o caráter coletivo da religião em se comparando com a magia que apesar de também conter um conjunto de crenças e ritos, não formam uma vida em sociedade, pública. O que evidencia esse aspecto coletivo da religião é a própria existência da igreja, local de reunião coletiva tanto de sacerdotes como de fiéis (p. 26-27).

Na conclusão do livro, Durkheim coloca que "quase todas as grandes instituições sociais nasceram da religião."
(p. 462)


\end{document}
