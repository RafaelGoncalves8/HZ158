\documentclass[12pt,a4paper]{article}
\setlength\parindent{24pt}
\usepackage[margin=2.5cm]{geometry}
\usepackage[utf8]{inputenc}
\usepackage[portuguese]{babel}
\renewcommand{\baselinestretch}{1.5}

\begin{document}

\noindent
\centerline{\large\textbf{HZ158 | Fichamento d'As formas elementares da vida religiosa}}
\centerline{\large Rafael Gonçalves (186062)}
\break
\noindent
O autor inicia a obra descrevendo seu objeto de pesquisa: as religiões das sociedades primitivas. Por mais que tenham significados diferentes, Durkheim usa as palavras \em primitivo\em , \em simples\em , \em elementar\em e \em original \em quase que como sinônimos ao longo do livro. Isso mostra uma característica importante do pensamento de Durkheim, a saber que as primeiras sociedades (originais, de onde se origina a vida em sociedade) seriam as sociedades mais simples, com menor grau de evolução (primitivas) e também elementares - as que conteriam os elementos essenciais para se entender as sociedades posteriores. Com isso podemos perceber como há no autor uma noção de evolução \em natural \em em uma direção específica que parte do simples, homogêneo e elementar rumo ao complexo, heterogêneo e diferenciado.

Dessa forma Durkheim busca analisar a vida religiosa por meio do estudo das sociedades em que ela se mostra da maneira mais simples, mas que ainda assim conteria os elementos para entender sua expressão mesmo nas sociedades de sua época tidas como mais complexas e evoluídas. Mais do que analisar a religião em sua forma primitiva, Durkheim busca analisar a própria origem da instituição religiosa, ou ainda a origem da própria sociedade (sociogênese).

Para o autor os conceitos essencias da filosofia e da ciência já estão presente nas representações religiosas. (falar mais sobre epistemologia e tempo, espaço, etc. religião como coisa social)

\end{document}
