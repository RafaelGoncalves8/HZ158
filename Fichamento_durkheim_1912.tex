\documentclass[12pt,a4paper]{article}
\setlength\parindent{24pt}
\usepackage[margin=2.5cm]{geometry}
\usepackage[utf8]{inputenc}
\usepackage[portuguese]{babel}
\renewcommand{\baselinestretch}{1.5}
\usepackage[bottom]{footmisc}

\begin{document}

\noindent
\centerline{\large\textbf{HZ158 | Fichamento d'As formas elementares da vida religiosa}}\footnote{DURKHEIM, Émile. 1996. As formas elementares da vida religiosa. (Trad. Paulo Neves) São Paulo: Martins Fontes [1912]}
\centerline{\large Rafael Gonçalves (186062)}
\break
\noindent
O autor inicia a obra expondo seu objeto de pesquisa: as religiões das sociedades primitivas (ou ainda, como o título explicita: as formas elementares da vida religiosa). Para analisar a instituição religiosa nessas sociedades, Durkheim faz uso de etnografias - que seriam o signo visível, concreto e analisável do objeto a ser estudado. Diferentemente de seus livros anteriores - muito mais focados nos modos de pensar e de agir da sociedade - os relatos etnograficos possibilitam a análise de uma esfera mais emocional, intensiva, ou os modos de sentir.

Por mais que tenham significados diferentes, Durkheim usa as palavras \em primitivo\em , \em simples\em , \em elementar \em e \em original \em quase que como sinônimos ao longo do livro. Isso mostra uma característica importante do pensamento durkheimiano, a saber que as primeiras sociedades (originais, de onde se origina a vida em sociedade) seriam as sociedades mais simples, com menor grau de evolução (primitivas) e também elementares - as que conteriam os elementos essenciais para se entender as sociedades posteriores. Com isso podemos perceber como há no autor uma noção de evolução \em natural \em em uma direção específica que parte do simples, primitivo, homogêneo e elementar rumo ao complexo, civilizado, heterogêneo e diferenciado.

Dessa forma, Durkheim busca analisar a vida religiosa por meio do estudo das sociedades em que ela se mostra da maneira mais simples, mas que ainda assim conteria os elementos para entender sua expressão mesmo nas sociedades de sua época tidas como mais complexas e evoluídas. Mas mais que analisar a religião em sua forma primitiva, Durkheim busca analisar a própria origem da instituição religiosa, ou ainda a origem da própria vida em sociedade (\em sociogênese\em ).

Para o autor, a religião é essencialmente \em coisa social\em .
``(...) as representações religiosas são representações coletivas que exprimem realidades coletivas; os ritos são maneiras de agir que só surgem no interior de grupos coordenados e se destinam a suscitar, manter ou refazer alguns estados mentais desses grupos." (p. XVI).
Ao tentar entender a origem da vida religiosa - primeira instituição cronologicamente, mas também de onde partiriam as outras -, Durkheim busca entender a origem da vida em comunidade. Como os indivíduos se associam a fim de formar uma sociedade.

Então o autor segue fazendo uma definição mais precisa do que seria a religião. Durkheim argumenta que para definir a religião é preciso definir seu sistema de representações (crenças) e suas práticas (ritos) (p. 19) . Todo conjunto de crenças parte da cassificação das coisas em dois gêneros opostos, de naturezas distintas, que Durkheim generaliza nos termos \em sagrado \em e \em profano\em . A religião nasce da experiência do sagrado, momento de comunhão, associação de indivíduos, nascimento do sagrado. Logo, em contraposição, aparece também o profano, o não-sagrado, o trivial, o puramente individual.
\begin{quote}
    `` (...) não nos resta outra coisa para definir o sagrado em relação ao profano, a não ser sua heterogeneidade. E o que torna essa heterogeneidade suficiente para caracterizar semelhante classificação das coisas e distingui-la de qualquer outra é justamente o fato de ela ser muito particular: \em ela é absoluta\em ." (p. 21-22)
\end{quote}
Ante a essa classificação, \em os indivíduos se unem em favor do sagrado\em , tentam repetir a experiência do mesmo, institucionalizá-la. São criadas crenças que serão representações que exprimem a natureza do sagrado e sua relação com o profano, e também ritos que serão os modos de agir diante das coisas sagradas. Esse sistema de crenças e ritos é o que constitue uma religião (p. 24).

É enfatizado o caráter coletivo da religião em se comparando com a magia que, apesar de também conter um conjunto de crenças e ritos, não formam uma vida em sociedade, pública. O que evidencia esse aspecto coletivo da religião é a própria existência da igreja, local de reunião coletiva tanto de sacerdotes como de fiéis (p. 26-27).

%falar sobre culto e prova experimental da crença, se der tempo (p. 460)

Na conclusão do livro, Durkheim coloca que ``(...) quase todas as grandes instituições sociais nasceram da religião."
(p. 462)
A religião conteria os elementos essenciais dos quais se derivaram as categorias científicas, as crenças e práticas mágicas, as regras da moral e do direito. Isso se dá porque o surgimento da religião coincide com o surgimento da sociedade. A religião só pode ser a origem das outras instituições, como argumenta Durkheim, pois ela contém os elementos essenciais da vida comum. É a expressão da associação de indivíduos em coletivo. ``(...) a ideia de sociedade é a alma da religião." (p. 462)

Da mesma forma que o surgimento da religião pode ser explicado como criação natural proveniente da intensidade ou da efervescência da vida coletiva - da experiência do sagrado -, a noção de sociedade ideal é um produto natural da vida social. O ideal não é algo exterior e antagônico ao real, mas sim parte do real. 
``A sociedade ideal não está fora da sociedade real, faz parte dela.”
(p.467)
Uma sociedade é composta por humanos, pelo espaço que ocupam, pelas técnicas que desenvolvem, por suas práticas, mas também por seu conjunto de ideias - sobretudo a ideia que faz de si mesma. Em outras palavras, a concepção de sociedade ideal, tal como está presente no pensamento coletivo, o \em ideal social\em, é produto das próprias condições sociais. Daí cada indivíduo imprime suas marcas pessoais ao desenvolver sua versão de sociedade ideal. Mas o \em ideal pessoal \em pode ser explicado pelo ideal social, deriva dele.

Assim como existe uma diferença entre o que seria o ideal social ou pessoal, Durkheim comenta sobre os conceitos, dizendo que
``
(...) as representações coletivas são mais estáveis que as individuais (...)
"
(p. 482)
pois as primeiras são fruto de uma consolidação no campo social. São impessoais e já instituídas na sociedade - inclusive na linguagem, pois ela própria seria um conjunto de conceitos - e, sendo assim, mudá-los requer uma ação contra o normal. Ao contrário, as representações individuais são facilmente moduladas por mudanças sejam elas externas ou internas ao indivíduo.

Ainda sobre os conceitos:

\begin{quote}
    ``
    Eles não são abstrações que só teriam realidade nas consciências particulares, mas representações tão concretas quanto as que o indivíduo pode ter de seu meio pessoal, representações que correspondem à maneira como esse ser especial, que é a sociedade, pensa as coisas de sua experiência própria.
    "
    (p. 483)
\end{quote}

Um conceito essencial é o de \em totalidade\em. (p. 490) Pois, se a função das categorias é envolver todos os outros conceitos, não existe categoria mais importante que a de totalidade. Mas só a sociedade engloba a visão genérica, não particular, mas representativa do todo, `` o conceito de totalidade não é senão a forma abstrata do conceito de sociedade: ela é o todo que compreende todas as coisas." (p. 491). A sociedade, neste sentido, é a consciência que está ao mesmo tempo acima e fora das representações individuais, abrange toda a realidade coletiva e descreve os aspectos permanentes e essenciais de todas as coisas.

É com base nisto também que Durkheim argumenta contra a dicotomia ciência-religião. Pois ambas partem da explicação do todo, do universal, do social. O desenvolvimento da ciência é, assim como outrora foi papel exclusivo das religiões, o desenvolvimento das representações coletivas.

\begin{quote}
``Sob esse aspecto [, o metodológico], ambas [ciência e religião] perseguem o mesmo objetivo: o pensamento científico é tão-só uma forma mais perfeita do pensamento religioso. Parece natural, portanto, que o segundo se apague progressivamente diante do primeiro, à medida que este se torne mais apto a desempenhar a tarefa."
(p. 475-476)
\end{quote}

Assim como a religião estudada no livro, a primeira impressão, parecia não conter nada de científico, as religiões contemporâneas parecem existir separadas à ciência. Mas o estudo presente no livro mostra explicações científicas do que julgavamos pertencer à religião. Ao longo da história a ciência já ocupou o lugar das religiões na explicação dos fenômenos materiais e os avanços da psicologia mostram um progressivo avanço da ciência rumo a explicação dos fenômenos que concernem ao mundo das almas. Tudo indica que a ciência se estabelecerá soberana à religião.

A ciência progressivamente substitúi o saber especulativo religioso por seu sistema de representações mais exato. Mas existe ainda a esfera do agir, central à religião.

\begin{quote}
``(...) por mais importantes que sejam os empréstimos tomados das ciências constituídas, eles não poderiam ser suficientes pois a fé é, antes de tudo, um impulso a agir e a ciência, por mais longe que se lance, sempre permanece à distância da ação."
 (p. 477-478)
\end{quote}

Portanto para Durkheim as religiões continuam a existir com suas especulações, mas estas não devem negar ou se contrapor â ciência, mas sim se apoiar nela. O sistema de representações religioso já não é mais hegemônico. A ciência aparece como potência rival às religiões e para o autor o controle da ciência sobre a fé e a religião se tornará cada vez mais amplo e eficaz, com uma influência ilimitada. (p. 478)

\end{document}
